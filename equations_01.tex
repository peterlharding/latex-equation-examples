\documentclass[]{article}

\usepackage[utf8]{inputenc}
\usepackage{amsmath}
\usepackage{nccmath}

\usepackage{lipsum}  % For generating dummy text

%opening
\title{}
\author{}

\begin{document}

\maketitle

\begin{abstract}

\end{abstract}

\section{Equation Examples}



\huge{
	\[ E = mc^2  \]
}

\[ f^{\prime 2} \]

\[ {}^{\dagger} \]

\[ F(x_{1}, x_{2}, \dots, x_{n})  \]

\[ \int\limits_{-\infty}^{\infty} e^{-x^{2}} \, dx = \sqrt{\pi} \]





Arithmetic equations are typed with a dollar sign. For example, $a + b$, $a - b$, $-a$, $a / b$, $a b$. There are different forms for multiplication and division that are $a \cdot b$, $a \times b$, $a \div b$



\begin{equation} \label{eqn}
	\huge{E = {mc^2}}
\end{equation}

The equation \ref{eqn} states mass equivalence relationship.



\newpage



\begin{multline*}
	p(x) = x^8+x^7+x^6+x^5 \\ 
	- x^4 - x^3 - x^2 - x
\end{multline*}





\begin{align*}
	a+b        &     a-b              &  (a+b)(a-b)\\
	x+y         &  x-y   &  (x+y)(x-y)\\
	p+q   &  p-q          &  (p+q)(p-q)
\end{align*}



\section{Grouping and Centering Equations}



\begin{gather*} 
	(a+b)=a^2+b^2+2ab \\ 
	(a-b)=a^2+b^2-2ab
\end{gather*}



\section{Parentheses and Brackets}



\[ 
\left \{
\begin{tabular}{ccc}
	1 & 5 & 8 \\
	0 & 2 & 4 \\
	3 & 3 & -8 
\end{tabular}
\right \}
\]



\newpage


\[ \big( \Big( \bigg( \Bigg( \]
\[ \big] \Big] \bigg] \Bigg] \]
\[ \big\{ \Big\{ \bigg\{ \Bigg\{\ \]
\[ \big \langle \Big \langle \bigg \langle  \Bigg \langle \]
\[ \big \rangle \Big \rangle \bigg \rangle  \Bigg \rangle \]



\section{Operators}


\[
\sin^2(a)+\cos^2(a) = 1
\]

\[
\lim_{h \rightarrow 0 } \frac{f(x+h)-f(x)}{h}
\]

This operator changes when used alongside 

text \( \lim_{x \rightarrow h} (x-h) \).


\section{Fractions and Binomials}

\[
\binom{n}{k} = \frac{n!}{k!(n-k)!}
\]

\[ f(x)=\frac{P(x)}{Q(x)} \ \ \textrm{and} 
\ \ f(x)=\textstyle\frac{P(x)}{Q(x)} \]

\[ \frac{1+\frac{a}{b}}{1+\frac{1}{1+\frac{1}{a}}} \]


\section{Left alignment of multiline equations in LaTeX}

\begin{flalign}
	f(u) & =\sum_{j=1}^{n} x_jf(u_j)&\\
	& =\sum_{j=1}^{n} x_j \sum_{i=1}^{m} a_{ij}v_i&\\
	& =\sum_{j=1}^{n} \sum_{i=1}^{m} a_{ij}x_jv_i
\end{flalign}

% Remove numbering in multiline equations 1

\begin{flalign*}
	f(u) & =\sum_{j=1}^{n} x_jf(u_j)&\\
	& =\sum_{j=1}^{n} x_j \sum_{i=1}^{m} a_{ij}v_i&\\
	& =\sum_{j=1}^{n} \sum_{i=1}^{m} a_{ij}x_jv_i
\end{flalign*}

% Remove numbering in multiline equations 2

\begin{flalign}
	f(u) & =\sum_{j=1}^{n} x_jf(u_j)&\\ \nonumber
	& =\sum_{j=1}^{n} x_j \sum_{i=1}^{m} a_{ij}v_i&\\ \nonumber
	& =\sum_{j=1}^{n} \sum_{i=1}^{m} a_{ij}x_jv_i
\end{flalign}


% Align environment

\begin{align}
	f(u) & =\sum_{j=1}^{n} x_jf(u_j)\\
	& =\sum_{j=1}^{n} x_j \sum_{i=1}^{m} a_{ij}v_i\\
	& =\sum_{j=1}^{n} \sum_{i=1}^{m} a_{ij}x_jv_i
\end{align}


% Aligned environment

\begin{equation}
	\begin{aligned}
		f(u) & =\sum_{j=1}^{n} x_jf(u_j)\\
		& =\sum_{j=1}^{n} x_j \sum_{i=1}^{m} a_{ij}v_i\\
		& = \sum_{j=1}^{n} \sum_{i=1}^{m} a_{ij}x_jv_i
	\end{aligned}
\end{equation}

\newpage


\begin{align}
	x   & = \frac{-b\pm\sqrt{b^2-4ac}}{2a}          \label{eq1} \\
	\frac{d}{dx}r^n & = nx^{n-1}                                \label{eq2} \\
	\mathrm{Length} & = \int_{a}^{b}\sqrt{[f't]^2+[g't]^2}dt    \label{eq3}
\end{align}

or

\begin{fleqn}
	\begin{equation}\label{eq1}
		x=\frac{-b\pm\sqrt{b^2-4ac}}{2a}
	\end{equation}
	\begin{equation}\label{eq2}
		\frac{d}{dx}r^n=nx^{n-1}
	\end{equation}
	\begin{equation}\label{eq3}
		Length=\int_{a}^{b}\sqrt{[f't]^2+[g't]^2}dt
	\end{equation}
\end{fleqn}

\newpage


\lipsum[11]

\begin{equation}\label{eq1}
	x=\frac{-b\pm\sqrt{b^2-4ac}}{2a}
\end{equation}

\begin{equation}\label{eq2}
	\frac{d}{dx}r^n=nx^{n-1}
\end{equation}

\begin{equation}\label{eq3}
	Length=\int_{a}^{b}\sqrt{[f't]^2+[g't]^2}dt
\end{equation}

or     

\begin{gather}
	x=\frac{-b\pm\sqrt{b^2-4ac}}{2a}            \label{eq1} \\
	\frac{d}{dx}r^n=nx^{n-1}                    \label{eq2} \\
	Length=\int_{a}^{b}\sqrt{[f't]^2+[g't]^2}dt \label{eq3}
\end{gather}

\lipsum[11]

\newpage


\begin{fleqn}[2em]    
	
	\begin{gather}
		x=\frac{-b\pm\sqrt{b^2-4ac}}{2a}            \label{eq1} \\
		\frac{d}{dx}r^n=nx^{n-1}                    \label{eq2} \\
		Length=\int_{a}^{b}\sqrt{[f't]^2+[g^1t]^2}dt \label{eq3}
	\end{gather}
	
\end{fleqn}

\newpage


\lipsum[11]

\begin{flalign} \label{eq11}
	& x =\frac{-b±\sqrt{b^2-4ac}}{2a} & \\
	\label{eq2}
	& \frac{d}{dx}r^n =nx^{n-1} \\
	\label{eq6}
	& \text{Length} = \int_{a}^{b} \sqrt{{[f^{1}t]^2+[g^{1}t]^2}\,dt}
\end{flalign}

Or even with the $ = $ signs aligned:

\begin{flalign} \label{eqa12}
	x & =\frac{-b±\sqrt{b²-4ac}}{2a} & \\
	\label{eqa2}
	\frac{d}{dx}r^n & =nx^{n-1} \\
	\label{eqa6}
	\text{Length} & = \int_{a}^{b} \sqrt{{[f^1t]+[g^{1}t]}\,dt}
\end{flalign}

\newpage


\end{document}
